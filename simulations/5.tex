\documentclass{article}
\usepackage{amsmath, amssymb, graphicx}

\begin{document}

\title{Basic Lectures on Emerging Design and Informatics III\\
Report 5}

\author{Goran Topi\'{c}, 49-106312}
\maketitle

Steady 1-dimensional advection-diffusion equation $T(x)$:
\begin{align}
  u\frac{dT}{dx} = \frac{1}{Pe}\frac{d^2 T}{dx^2}
\end{align}

Given $u=1.0$,
\begin{align}
  \frac{dT}{dx} = \frac{1}{Pe}\frac{d^2 T}{dx^2}
\end{align}

The domain of $x$ is $[0.0, 1.0]$, and additionally these conditions are given:
\begin{align}
  \label{cond1} T(0.0) &= 1.0 \\
  \label{cond2} T(1.0) &= 0.0
\end{align}

Task 1: Analytic solution of the differential equation

Since this is a homogenous differential equation with constant coefficients, we can be certain that $T$ will have the form of $T(x) = e^{rx}$. If we make the substitution:
\begin{align}
  \frac{d(e^{rx})}{dx} &= re^{rx} \\
  \frac{d(e^{rx})}{dx} &= r^2 e^{rx}
\end{align}
\begin{align}
  re^{rx} &= \frac{1}{Pe} r^2 e^{rx} \\
  re^{rx} - \frac{1}{Pe} r^2 e^{rx} &= 0 \\
  e^{rx} r (1 - \frac{1}{Pe} r) &= 0
\end{align}

Since an exponential function never has roots,
\begin{align}
  r (1 - \frac{1}{Pe} r) &= 0
\end{align}

Thus, there are two solutions for r:
\begin{align}
  r_1 &= 0 \\
  1 - \frac{1}{Pe} r_2 &= 0 \\
  r_2 &= Pe
\end{align}

and consequently we can see two possible solutions of the differential equation:
\begin{align}
  T_1(x) &= e^{r_1 x} = 1 \\
  T_2(x) &= e^{r_2 x} = e^{Pe~x}
\end{align}

Since all solutions of a second-order homogeneous differential equation can be expressed as a linear combination of two particular, linearly independent solutions, and $T_1$ and $T_2$ are linearly independent, we can get a general solution in the following way:
\begin{align}
  T(x) &= a T_1(x) + b T_2(x) \\
       &= a + b e^{Pe~x}
\end{align}

Using the conditions from \eqref{cond1} and \eqref{cond2}, we get the following equation system:
\begin{align}
  1.0 &= a + b e^{0.0 Pe} \\
  0.0 &= a + b e^{1.0 Pe}
\end{align}

i.e.
\begin{align}
  1 &= a + b \\
  0 &= a + b e^{Pe} \\
  1 &= b - b e^{Pe} \\
    &= b (1 - e^{Pe}) \\
  b &= \frac{1}{1 - e^{Pe}} \\
  a &= 1 - b \\
    &= 1 - \frac{1}{1 - e^{Pe}}
\end{align}

Thus,
\begin{align}
  T(x) &= 1 - \frac{1}{1 - e^{Pe}} + \frac{1}{1 - e^{Pe}} e^{Pe~x} \\
       &= \frac{1}{1 - e^{Pe}}((1 - e^{Pe}) - 1 + e^{Pe~x})
\end{align}

and we finally obtain
\begin{align}
  T(x) &= \frac{e^{Pe~x} - e^{Pe}}{1 - e^{Pe}}
\end{align}

Task 2-1: Discretisation of the differential equation using the central difference method

\begin{align}
  \frac{dT}{dx} &= \frac{1}{Pe}\frac{d^2 T}{dx^2} \\
  \frac{T_{i + 1} - T_{i - 1}}{2 \Delta x} &= \frac{1}{Pe} \frac{T_{i + 1} - 2 T_i + T_{i - 1}}{(\Delta x)^2}
\end{align}

We can see that the next value $T_{i + 1}$ depends on the previous two values ($T_i$, $T_{i - 1}$), yet only $T_0$ and $T_N$ are given. Thus, we will convert the equation into a form we can use in Gauss-Seidel algorithm, using the central difference method:
\begin{align}
  \frac{dT}{dx} &= \frac{1}{Pe}\frac{d^2 T}{dx^2} \\
  \frac{T_{i + 1} - T_{i - 1}}{2 \Delta x} &= \frac{1}{Pe} \frac{T_{i + 1} - 2 T_i + T_{i - 1}}{(\Delta x)^2}
\end{align}
\begin{align}
  T_{i - 1}(\frac{1}{Pe} \frac{1}{(\Delta x)^2} + \frac{1}{2 \Delta x}) - T_i \frac{1}{Pe} \frac{2}{(\Delta x)^2} + T_{i + 1}(\frac{1}{Pe} \frac{1}{(\Delta x)^2} - \frac{1}{2 \Delta x}) &= 0
\end{align}

Task 3-1: Discretisation of the differential equation using the forward difference method for the first-order differential, and the central difference method for the second-order differential

\begin{align}
  \frac{T_{i + 1} - T_{i}}{\Delta x} &= \frac{1}{Pe} \frac{T_{i + 1} - 2 T_i + T_{i - 1}}{(\Delta x)^2}
\end{align}
\begin{align}
  T_{i - 1}(\frac{1}{Pe} \frac{1}{(\Delta x)^2} + \frac{1}{2 \Delta x}) - T_i \frac{1}{Pe} \frac{2}{(\Delta x)^2} + T_{i + 1}(\frac{1}{Pe} \frac{1}{(\Delta x)^2} - \frac{1}{2 \Delta x}) &= 0
\end{align}


\end{document}
