\documentclass{article}
\usepackage{amsmath, amssymb, graphicx}

\begin{document}

\title{Basic Lectures on Emerging Design and Informatics III\\
Report 4}

\author{Goran Topi\'{c}, 49-106312}
\maketitle

The original differential function:
\begin{align}
  \frac{dy}{dx} &= \frac{1 + y^2}{1 + y^2} \\
  \frac{dy}{1 + x^2} &= \frac{dx}{1 + y^2} \\
  \int\frac{dy}{1 + x^2} &= \int\frac{dx}{1 + y^2} \\
  \arctan x + C &= \arctan y \\
  y &= \tan(\arctan x + C)
\end{align}

Using the starting values, we can find the concrete solution. We can
ignore the cyclic nature of $tan$, as it does not impact the solution in
any way.
\begin{align}
  x_0 &= 0 \\
  y_0 &= 2 \\
  \arctan 0 + C &= \arctan 2 \\
  C &= \arctan 2 \\
  y &= \tan(\arctan x + \arctan 2)
\end{align}

Since the function is discontinuous and has a pole at $x = 0.5$, this
report approximates the function at $\max x_n, x_n \leqslant 0.4$. The
errors at the high end of $h$ occur because $x_n$ is more likely to end
up further from $0.4$ (as $0.4$ is not cleanly divisible by most of the
values of $h$), and the error consequently ends up being measured on a
smaller interval.

\end{document}
